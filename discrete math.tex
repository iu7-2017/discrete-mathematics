\documentclass[]{article}
\usepackage[utf8]{inputenc}
\usepackage[russian]{babel}
\usepackage{ amssymb }
\usepackage{color}

%opening
\title{}
\author{}

\begin{document}

\textbf{МОДУЛЬ 1: Множества, отношения, алгебры}


Вопросы для подготовки к рубежному контролю

\bigskip

\begin{enumerate}
\item % -----------------------------------1---------------------------------------
\textbf {Множества, подмножества. Способы определения множеств. Равенство множеств. Операции над множествами (объединение, пересечение, разность, симметрическая разность, дополнение). Методы доказательства теоретико-множественных тождеств.}

\hrulefill

Исходное не определяемое строго понятие\\\\
Определение по Кантору: Под многообразием или множеством я понимаю вообще всё многое, которое возможно мыслить как единое, т.е. такую совокупность определяемых элементов, которая посредством одного закона может быть соеденена в одно целое.

Может быть записано, как $A = \{x: P(x)\}$, где P -- характеристический предикат

\textbf{Равенство множеств}
$A = B \rightleftharpoons (x \in A \iff x \in B)$

\textbf{Подмножество} $A \subseteq B \rightleftharpoons (\forall x) (x \in A \Rightarrow x \in B)$

\textbf{Строгое подмножество} $A \subset B \rightleftharpoons (A \leq B) \land (A \neq B)$

Операции

\textbf{Объединение} $A \cup  B = \{x: (x \in A) \lor (x \in B) \}$

\textbf{Пересечение} $A \cap B = \{x: (x \in A) \land (x \in B) \}$

\textbf{Разность} $A \setminus B = \{x: (x \in A) \land (x \notin B) \}$ 

\textbf{Симметрическая разность} $A \bigtriangleup B = (A \setminus B) \cup (B \setminus A)$

\textbf{Дополнение} $\overline{\rm A} = \{x: x \notin A\}$

Методы доказательств:  

Метод двух включений

Метод характеристических функций

Метод эквивалентных преобразований

\hrulefill
\item % -----------------------------------2---------------------------------------
\textbf{Неупорядоченная пара, упорядоченная пара, кортеж. Декартово произведение множеств.}

\hrulefill

\textbf{Неупорядоченная пара} на множествах $A$ и $B$ 

$A,B \neq \emptyset \quad a \in A, b \in B \quad \{a,b\}$

\textbf{Упорядоченная пара} $(a,b)$ определяется не только элементами $a$ и $b$, но и порядком в котором они записаны 

\textbf{Кортеж} $(a_1,a_2,...,a_n)$ на множествах $A_1,A_2,...,A_n$ -- обобщение пары для $n$ элементов

Множество всех кортежей длины $n$ на множествах $A_1,A_2,...,A_n$ называют \textbf{декартовым произведением} множеств $A_1,A_2,...,A_n$ и обозначают  $A_1 \times A_2 \times ... \times A_n$.

\hrulefill

\item % -----------------------------------3---------------------------------------
\textbf{Отображения: область определения, область значений. Инъективное, сюръективное и биективное отображения. Частичное отображение.}

\hrulefill

$f: A \rightarrow B$ -- отображение $A$ в $B$

{\color{red} Пусто}

\hrulefill
\item % -----------------------------------4---------------------------------------
\textbf{Соответствия. График и граф соответствия, область определения, область значения. Сечение соответствия. Сечение соответствия по множеству. Функциональность соответствия по компоненте. Бинарные и n-арные отношения. Связь между отношениями, соответствиями и отображениями.}

\hrulefill

{\color{red} Пусто}

\hrulefill
\item % -----------------------------------5---------------------------------------
\textbf{Композиция соответствий, обратное соответствие и их свойства (с доказательством).}

\hrulefill

\textbf{Композицией соответствий} 
$\rho \subseteq A \times B, \sigma \subseteq B \times C$ 
называют соответствие 
$\rho \circ \sigma = \{ (x,y): (\exists z \in B) ((x,z) \in \rho) \land ((z,y)) \in \sigma \} $.

Cвойства:

1) $\rho \circ (\sigma \circ \tau) = (\rho \circ \sigma) \circ \tau $ (Ассоциативность)

Доказательство:

\begin{math}
(x,y) \in \rho \circ (\sigma \circ \tau) \Rightarrow\\
(\exists z) ((x,z) \in \rho) \land ((z,y) \in (\sigma \circ \tau)) \Rightarrow \\
(\exists z,u) ((x,z) \in \rho) \land ((z,u) \in \sigma) \land ((t,y) \in \tau) \Rightarrow \\
(\exists u)((x,u) \in (\rho \circ \sigma)) \land ((u,y) \in \tau) \Rightarrow\\
(x,y) \in (\rho \circ \sigma) \circ \tau
\end{math}

Обратно аналогично


2) $ \rho \circ (\sigma \cup \tau) =  (\rho \circ \sigma) \cup (\rho \circ \tau)$ (Дистрибутивность относительно объединения)

Доказательство:

\begin{math}
(x,y) \in \rho \circ (\sigma \cup \tau) \Rightarrow\\
(\exists z) ((x,z) \in \rho) \land ((z,y) \in (\rho \cup \tau)) \Rightarrow\\
(\exists z) ((x,z) \in \rho) \land (((z,y) \in \rho) \lor ((z,y) \in \tau)) \Rightarrow\\
(\exists z) (((x,z) \in \rho) \land ((z,y) \in \sigma)) \lor (((x,z )\in \rho) \land ((z,y) \in \tau)) \Rightarrow\\
(x,y) \in \rho \circ \sigma \lor (x,y) \in \rho \circ \tau \Rightarrow\\
(x,y) \in \rho \circ \sigma \cup \rho \circ \tau
\end{math}

Обратно 

\begin{math}
(x,y) \in \rho \circ \sigma \cup \rho \circ \tau \Rightarrow\\ 
(\exists u) (((x,u) \in  \rho) \land ((u,y) \in \sigma)) \lor (\exists v)(((x,v) \in  \rho) \land ((u,v) \in \tau)) \Rightarrow\\ 
(\exists z) (((x,z) \in \rho) \land (((z,y) \in \sigma) \lor ((z,y) \in \tau))) \Rightarrow\\ 
(\exists z) (((x,z) \in \rho) \land ((z,y) \in \sigma \circ \tau)) \Rightarrow\\ 
(x,y) \in \rho \circ (\sigma \cup \tau)
\end{math}

{\color{red} Доказательства двух включений не симметричны! u и v во второй половине не обязанны быть равны}

3) $\rho \circ \emptyset = \emptyset \circ \rho = \emptyset$

{\color{red} Нет доказательства}

4) Для $\rho \subseteq A^2: \rho \circ id_a = id_a \circ \rho = \rho $




{\color{red} Нет доказательства}

\textbf{Обратным соответствием} к соответствию $\rho \in A \times B$ называют соответствие
$\rho^{-1} = \{(y,x): (x,y) \in \rho\} $

Свойства:

1)  $(\rho^{-1})^{-1} = \rho$

{\color{red} Нет доказательства}

2) $(\rho \circ \sigma)^{-1} = \rho^{-1} \circ \sigma^{-1}$


{\color{red} Нет доказательства}


\hrulefill
\item % -----------------------------------6---------------------------------------
\textbf{Специальные свойства бинарных отношений на множестве (рефлексивность, иррефлексивность, симметричность, антисимметричность, транзитивность).}

\hrulefill

\textbf{Рефлексивность} $\forall x : x\rho x$

\textbf{Иррефлексивность} $\nexists x:  x\rho x $

\textbf{Симметричность} $\forall x,y : (x \rho y \Rightarrow y \rho x) $

\textbf{Антисимметричность} $\forall x,y : (x\rho y) \land (y \rho x) \Rightarrow x = y $

\textbf{Транзитивность} $\forall x,y,z : (x\rho y) \land (y \rho z) \Rightarrow x\rho z $

\hrulefill


\item % -----------------------------------7---------------------------------------
\textbf{Классификация бинарных отношений на множестве: эквивалентность, толерантность, порядок, предпорядок, строгий порядок.}

\hrulefill

Бинарное отношение на некотором множестве называют 

\textbf{Эквивалентностью}, если оно рефлексивно, симметрично и транзитивно

\textbf{Толерантностью}, если оно рефлексивно и симметрично

\textbf{Порядком}, если оно рефлексивно, антисимметрично и транзитивно

\textbf{Предпорядком}, если оно рефлексивно и транзитивно

\textbf{Строгим порядком}, если оно иррефлексивно, антисимметрично и транзитивно


\hrulefill
\item % -----------------------------------8---------------------------------------
\textbf{Отношение эквивалентности. Класс эквивалентности. Фактор-множество.}

\hrulefill

{\color{red} Пусто}

\hrulefill
\item % -----------------------------------9---------------------------------------
\textbf{Отношения предпорядка и порядка. Наибольший, максимальные, наименьший и минимальные элементы. Точная нижняя и верхняя грани множества.}

\hrulefill

{\color{red} Пусто}

\hrulefill
\item % -----------------------------------10--------------------------------------
\textbf{Точная верхняя грань последовательности. Индуктивное упорядоченное множество. Теорема о неподвижной точке (с доказательством). Пример вычисления неподвижной точки.}

\hrulefill

{\color{red} Пусто}

\end{enumerate}



$\forall x \in X, \quad \exists y \leq \epsilon$

\section{}

\end{document}
